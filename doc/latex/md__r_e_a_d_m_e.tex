\#nann

\subsection*{nann Python接口}

在$\ast$$\ast$nann$\ast$$\ast$目录中使用以下命令进行安装 
\begin{DoxyCode}
1 python setup.py build
2 sudo python setup.py install
\end{DoxyCode}
 安装完成后,在\+\_\+python\+\_\+代码中使用{\ttfamily import nann},即可使用。

\subsubsection*{接口列表}

$\vert$接口名$\vert$参数$\vert$参数说明$\vert$返回值$\vert$说明$\vert$ $\vert$---$\vert$---$\vert$---$\vert$---$\vert$---$\vert$ $\vert$load$\vert$无$\vert$错误代码$\vert$加载nann库,在最之前调用$\vert$ $\vert$unload$\vert$无$\vert$无$\vert$错误代码$\vert$卸载nann库,在进程结束时调用,与load成对使用$\vert$ $\vert$create$\vert$task$\vert$任务名(I\+D),用来标示任务$\vert$当前创建管理器的总数$\vert$创建一个nann管理器,此管理器只训练任务task$\vert$ $\vert$$\vert$$\ast$$\ast$create.json$\ast$$\ast$$\vert$一段json代码,用于描述神经网络,详细参见后面的章节$\vert$$\vert$$\vert$ $\vert$$\vert$max\+\_\+calc$\vert$容纳计算结点(线程)最大数量$\vert$$\vert$$\vert$ $\vert$$\vert$now\+\_\+calc$\vert$当前要启动的计算结点(可选参数)$\vert$$\vert$$\vert$ $\vert$destroy$\vert$task$\vert$任务名$\vert$错误代码$\vert$销毁与task关联的nann管理器$\vert$ $\vert$exptype$\vert$exp$\vert$1\+:抛出异常,0\+:返回错误代码$\vert$无$\vert$修改异常处理类型$\vert$ $\vert$training$\vert$task$\vert$任务名$\vert$错误代码$\vert$有目标训练,进入回调函数,调整权值矩阵,输出结果$\vert$ $\vert$$\vert$$\ast$$\ast$input.json$\ast$$\ast$$\vert$一段json代码,用于描述输入,详细参加后面的章节$\vert$$\vert$$\vert$ $\vert$training\+\_\+notarget$\vert$task$\vert$任务名$\vert$错误代码$\vert$无目标训练,进入回调函数,不调整权值矩阵,输出结果,仅用来做分类计算$\vert$ $\vert$$\vert$$\ast$$\ast$input.json$\ast$$\ast$$\vert$一段json代码,用于描述输入,详细参加后面的章节$\vert$$\vert$$\vert$ $\vert$training\+\_\+nooutput$\vert$task$\vert$任务名$\vert$错误代码$\vert$有目标训练,进入回调函数,调整权值矩阵,不输出结果,仅用来做训练$\vert$ $\vert$nnn\+\_\+read$\vert$filename$\vert$文件路径$\vert$失败\+:0成功\+:1$\vert$读取一个nnn文件到内存中,可直接使用任务名进行计算,$\ast$$\ast$当前版本不支持$\ast$$\ast$$\vert$ $\vert$nnn\+\_\+write$\vert$task$\vert$任务名$\vert$错误代码$\vert$写入一个任务的神经网络到nnn文件中,$\ast$$\ast$当前版本不支持$\ast$$\ast$$\vert$ $\vert$$\vert$filename$\vert$文件名$\vert$$\vert$$\vert$ $\vert$test$\vert$无$\vert$无$\vert$无$\vert$测试是否工作正常,无实际意义$\vert$ $\vert$version$\vert$无$\vert$无$\vert$版本字符串$\vert$返回当前nann版本(按圆周率计算)$\vert$ $\vert$print$\vert$task$\vert$任务名$\vert$错误代码$\vert$打印当前任务的ann内容到控制台$\vert$ $\vert$iscalcing$\vert$task$\vert$任务名$\vert$任务点数量$\vert$还有多少正在进行运算的任务结点$\vert$ $\vert$merge$\vert$task$\vert$任务名$\vert$错误代码$\vert$合并当前计算完毕的任务结点的ann为一个$\vert$

\#\#\# 使用说明 
\begin{DoxyCode}
1 import nann
2 
3 print nann.version()  # 不加载也可以使用
4 
5 nann.load()
6 
7 # 修改异常类型为不抛出异常,由自己控制错误代码
8 nann.exptype(0)
9 # 接收任务
10 if (nann.create(task, nann\_json) != 0):
11     print "创建错误"
12 
13 if (nann.training(task, samples\_json) != 0):
14     print "训练错误"
15 
16 # ... 其余的代码
17 
18 # 销毁
19 nann.destroy(task)
20 
21 # 可创建其他的任务结点
22 
23 nann.unload()
\end{DoxyCode}
 \subsubsection*{错误代码}

\begin{TabularC}{3}
\hline
\rowcolor{lightgray}{\bf 错误代码}&{\bf 说明}&{\bf 级别  }\\\cline{1-3}
0x88000000&内部错误&错误 \\\cline{1-3}
0x88000001&分析\+J\+S\+O\+N文件错误&错误 \\\cline{1-3}
0x88000100&参数错误&错误 \\\cline{1-3}
0x88000101&任务已经存在&错误 \\\cline{1-3}
0x88000102&任务不存在&错误 \\\cline{1-3}
0x88000200&分配内存失败&错误 \\\cline{1-3}
0x87000000&内部警告&警告 \\\cline{1-3}
0x87000100&参数错误&警告 \\\cline{1-3}
0x87000101&任务已经存在&警告 \\\cline{1-3}
0x87000102&任务不存在&警告 \\\cline{1-3}
\end{TabularC}
其中$\ast$$\ast$警告$\ast$$\ast$的意思是,不影响当前运行,但是可能会发生意外。$\ast$$\ast$错误$\ast$$\ast$级别直接退出运行。

\subsubsection*{json输入详解}

在以下两份json中,都存在$\ast$$\ast$target$\ast$$\ast$一项,其中input.json中的$\ast$$\ast$target$\ast$$\ast$会覆盖掉create.json中的$\ast$$\ast$target$\ast$$\ast$。后者为全局配置目标。在每次实际训练中调节。

\#\#\#\# create.\+json 
\begin{DoxyCode}
1 \{
2     "alg": "rcalc",
3     "ann": \{
4         "weight matrixes": \{
5             "0": \{
6                 "r1": [0.75, 0.83, 0.39],
7                 "r2": [0.98, 0.43, 0.12],
8                 "r3": [0.12, 0.45, 0.78],
9                 "r4": [0.11, 0.12, 0.65],
10                 "r5": [0.56, 0.67, 0.34]
11             \},
12             "1": \{
13                 "r1": [0.35, 0.23, 0.19, 0.37],
14                 "r2": [0.51, 0.49, 0.72, 0.11],
15                 "r3": [0.24, 0.31, 0.71, 0.51]
16             \},
17             "2": \{
18                 "r1": [0.15, 0.25, 0.41],
19                 "r2": [0.21, 0.29, 0.25],
20                 "r3": [0.34, 0.11, 0.87],
21                 "r4": [0.61, 0.92, 0.93]
22             \}
23         \},
24         "delta weight matrixes": \{
25             "0": \{
26                 "r1": [0.07, 0.08, 0.09],
27                 "r2": [0.09, 0.04, 0.02],
28                 "r3": [0.01, 0.04, 0.08],
29                 "r4": [0.01, 0.02, 0.05],
30                 "r5": [0.05, 0.07, 0.04]
31             \},
32             "1": \{
33                 "r1": [0.05, 0.03, 0.09, 0.07],
34                 "r2": [0.01, 0.09, 0.02, 0.01],
35                 "r3": [0.04, 0.01, 0.01, 0.01]
36             \},
37             "2": \{
38                 "r1": [0.05, 0.05, 0.01],
39                 "r2": [0.01, 0.09, 0.05],
40                 "r3": [0.04, 0.01, 0.07],
41                 "r4": [0.01, 0.02, 0.03]
42             \}
43         \}
44     \},
45     "target": [0.11, 0.23, 0.78]
46 \}
\end{DoxyCode}
 \paragraph*{input.\+json}


\begin{DoxyCode}
1 \{
2     "samples": \{
3         "t1": [0.1, 0.2, 0.34, 0.45, 0.55],
4         "t2": [0.21, 0.6, 0.4, 0.23, 0.51],
5         "t3": [0.31, 0.7, 0.31, 0.1, 0.46],
6         "t4": [0.41, 0.9, 0.21, 0.61, 0.78]
7     \},
8     "target": [0.11, 0.23, 0.78]
9 \}
\end{DoxyCode}
 